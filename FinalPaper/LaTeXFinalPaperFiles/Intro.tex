\section{Introduction}

This document describes the work put in to create a program that runs in Unity to model traffic in 3D.
It allows for users to determine a start position, as well as a desired ending position which will follow traffic rules to ensure a fast and legal guide.

The code is able to be viewed on Github at this \href{https://github.com/EZRA-DVLPR/TrafficModeling}{Link}.

\subsection{Background}

Traffic congestion is a very common experience for those who live in Los Angeles \cite{CNN}.
In order to address this problem many apps such as Waze and Google Maps were created \cite{TraffApps}.
Following in the style of these applications, we wanted to create software that follows a similar purpose: to provide a good model for traffic flow and path to a destination with this information.

\subsection{Related Works}

We were interested in trying to model a similar outcome as outlined in this paper \cite{TrafficPaper}.
In order to achieve this, we would utilize Unity, a 3rd party package called EasyRoads \cite{EasyRoads}, MLAgents \cite{MLAgents}, and an external source for the traffic data.
To obtain the traffic data, we would utilize real-time traffic information from TomTom \cite{TomTom}.
We initially were hoping to utilize ML Learning Agents to do something similar to what was done in this Youtube video \cite{AITrackmania}.
In the end we were decided against implementing ML Learning Agents for our project, as we would utilize Unity's built-in pathfinding capabilities.

\subsection{Stakeholders}

The Major Stakeholders are: 
\begin{enumerate}
    \item Users: they want a product that will work by giving accurate and detailed information of traffic.
    Additionally, they want the data to be useful for determining the route to take from their source destination to the desired ending destination.
    
    \item Developers: they want to have a simplistic model for creating, modifying, and debugging the project across its various languages, scripts, and so forth.
    Development would be made easy by utilizing professional tools, technologies, and features.
\end{enumerate}

\subsection{Document Structure}

This document contains a series of 3 main sections:
\begin{enumerate}
    \item Design Goals: This section outlines the main objectives and guiding principles that drive the design of the software project. 
    It provides a high-level vision of what the software aims to achieve in terms of functionality, usability, performance, and other key aspects.
    
    \item System Behavior: This section describes how the system behaves as a whole and its overall functionality. 
    It should give readers a clear understanding of the system's expected actions and reactions in different scenarios.
    
    \item Logical View: The logical view provides a conceptual model of the software's architecture and its components. 
    This includes how the software is organized and how its different parts work together.

    \item Scenario View: The scenario view illustrates how the system behaves in specific use cases or scenarios. 
    It helps to show how the system meets the requirements in practice.
\end{enumerate}

We will explain the project utilizing UML diagrams, User-Interaction Diagrams, Class Diagrams, and other commonly used professional Software Development Models.
This is to ensure that out project is explained with the utmost clarity, and conciseness.

\subsection{Responsibilities}

Isaiah: API connectivity.
Obtained the Traffic Data and Map Data through the use of a Python Script and TomTom's developer API.
Added ML Agents to the unity project.

Jae: 

Anastasia: User Interface, Pathfinding, Car spawning, Unity implementation